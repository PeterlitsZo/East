\section{示例}

[ 需要更新 ]

\begin{lstlisting}
$ # ---[ build itself ]------------------------------------------
$
$ go build -o main ./src
$ ./main --help
usage: East [-h|--help] [-d|--dirpath "<value>"] [-c|--command "<value>"]
            [-m|--mkindex] [-u|--useindex] [-i|--interactive]

            sreach engine on file system

Arguments:

  -h  --help         Print help information
  -d  --dirpath      the input files' folder path. Default: input
  -c  --command      the command to get the ID list (see README.pdf). Default: 
  -m  --mkindex      use this flag to make index named 'index.dict'
  -u  --useindex     use file 'index.dict' to find result
  -i  --interactive  make self under the interactive mode
$
$ # ---[ a little test using `-command` ]------------------------
$
$ ./main --command="'in' || 'not'"
result: [ d01.txt, d10.txt, d02.txt, d03.txt, d04.txt, d06.txt, d07.txt, d08.txt, d09.txt ]
$
$ # -[ use flag `-mkindex` and `useindex` to hold the result ]---
$
$ ls
README.md
README.pdf
input
main
make.sh
src
$
$ ./main --mkindex
$
$ ls # it will make a new file named index.dict
README.md
README.pdf
index.dict
input
main
make.sh
src
$
$ ./main --useindex --command="not 'in'"
result: [ d5.txt ]
$
$ ---[ the interactive mode ]------------------------------------
$
$ ./main --useindex --interactive
Enter `quit` for quit
copyleft (C) Peterlits Zo <peterlitszo@outlook.com>
Github: github.com/PeterlitsZo

Command > "that" and 'peter'
result: [ ]
Command > 'that' and ('peter' or !'peter')
result: [ d06.txt, d07.txt ]
Command > quit
$ 
$ 
\end{lstlisting}

