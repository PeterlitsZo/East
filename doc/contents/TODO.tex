\section{TODO、版本历史等辅助信息}

\subsection{ToDo}
\begin{plttodoenv}{1}
    \t[ ] 将name$\to$content的数据处理为文档集的df-idf结构。
    \t[ ] 有什么办法可以来一键安装\verb|GoLang|的依赖吗?
    \t[ ] 更加详细的,函数式的结构。
    \t[ ] 可以储存index的信息。
    \t[ ] 命令\verb|useindex|和\verb|mkindex|。
    \t[v] 定义文档集的数据结构。(verb|map[string]string|)
    \t[v] 设计一个函数,对指定的字符串分词形成向量(类型:词典)。
    \t[v] 指出list来列出所有文件。
\end{plttodoenv}
\bigskip

\subsection{版本历史}

\hline

0.4.6: 因为在上一版本中的支持,很轻松就实现了空命令:即,什么都他妈的不干。
此外,实现了列引索命令\verb|list|和重构了\verb|sreach|。现在v0.3.0版本能干
的,这个版本都能干,这个版本能的,那个版本大多数不能。这个小版本就差不多到
此为止了。重构了解析和处理层,接下来就是df-idf的内容攻坚了。

0.4.5: 通过回车来分割命令。原来在\verb|unicode.IsSpace|的眼中,回车和其他的
空白符号都是一样的,原来如此,让我搞得好辛苦。

0.4.4: 支持更好的\verb|quit|,与之前不同,\verb|quit|是一个命令而不需要预处理了。

0.4.3: 支持一个没有什么用的命令:\verb|print|,还有,去你妈的默认选项。
没有默认选项了现在。

0.4.2: 全面更新了文档。使用\LaTeX{}而非\verb|Markdown|来编写文档。

0.4.1: 重构,并让\verb|interactive|成为默认选项。

0.4.0: 使用\verb|sub-command|。

\hline

0.3.0: 使用前置命令以支持更多的操作,目前支持命令\verb|sreach|。

\hline

0.2.4: 使用\verb|Peterlits/argparse|替代原作者的库,以获得更好的usage输
出(不过如果原作者如果接受了我的pull request的话,那么其实可能又
会换回来)。

\hline

before: 未记录。

