\section{部署与使用}

\subsection{部署}

请按照:
\begin{lstlisting}
$ go get -u -v github.com/PeterlitsZo/argparse
$ go build -o main ./src
\end{lstlisting}

或者:(如果在Linux环境下推荐使用)
\begin{lstlisting}
$ make init
$ make
\end{lstlisting}

进行编译。因为本软件是基于Goolge开发的开源语言\go 编写而成,使用
前需要安装\go 的编译器。\go 语言是一门比较小众的语言,所以如果没有
该编译器的话,还请进入\url{https://golang.org/dl/}下载。

输入命令\verb|./main version|来检查是否部署成功。如果部署成功,则会出现
如下的信息:
\begin{lstlisting}
$ ./main version
East version 0.4.x
$
\end{lstlisting}

%%%%%%%%%%%%%%%%%%%%%%%%%%%%%%%%%%%%%%%%%%%%%%%%%%%%%%%%%%%%%%%

\subsection{入门}

\subsubsection{主要命令}

编译完成后,输入下命令可以得到主要用法:
\begin{lstlisting}
$ ./main --help
usage: East <Command> [-h|--help]
description:
  sreach engine on file system
commands:
  run          the command to get the ID list (see README.pdf)
  mkindex      use this flag to make index named 'index.dict'
  interactive  make Self under the Interactive mode
  version      Show East's Version
arguments:
  -h, --help  Print help information
$
\end{lstlisting}

通过\verb|mkindex|命令可以为\verb|dirpath|文件夹(默认值为``input'',但可以根
据\verb|--dirpath|来进行修改)生成引索,然后通过\verb|--useindex|来让其
它命令使用引索文件。如果没有\verb|--useindex|命令,则会动态遍历
文件夹,分词,以哈希表加链表作为数据结构(当然,使用引索文件的
话,虽然不需要遍历分词,但是还是需要抽象成数据结构)。

最后使用\verb|run|命令或者进入\verb|interactive|模式来完成主要目标:检索。
语法见后文。

\subsubsection{检索语言}

这一次选择使用\verb|goyacc|工具来生成相应的LALR解析器源码,我首先实现了
一个分词器,然后编写BNF式的语法规则,构建了解析器\verb|parse.y|。

语法规则如下:
\begin{lstlisting}
<ast>          ::= SREACH <sreach_word>
                |  LIST
<sreach_word>  ::= <expr> OR <sreach_word>
                |  <expr>
<expr>         ::= <atom> AND <expr>
                |  <atom>
<atom>         ::= NOT STR
                |  STR
                |  NOT '(' <sreach_word> ')'
                |  '(' <sreach_word> ')'
\end{lstlisting}

实例:
\begin{lstlisting}
1. sreach 'in' AND NOT 'in' OR 'her'
2. sreach "'" or not '"'
3. sreach 'outside' && !'inside'
4. sreach (('that' or !'that') and 'that')
\end{lstlisting}

如实例所述,STR字符串是由单引号或者双引号包裹而成,支持为单引号或
者双引号转义(其实在命令行结构下本身就需要对引号进行转义,所以难免
会发生二次转义,比如\verb|--command="'\\\''"|这样比较丑陋的语法)。

不过现在甚至可以在交互模式下使用了。使用\verb|-interactive|进入交互模式
,然后在交互下输入命令,就算是同时使用单引号和双引号也轻轻松松、再
也不用担心二次转义啦(看到就是赚到)。

而AND,OR,NOT不仅仅支持单词(忽略大小写)样式,还支持C式的逻辑
运算符,也就是说\verb|&&|,\verb-||-和\verb|!|这三个传统命令。

现在还支持括号来实现更加高级的操作了。

