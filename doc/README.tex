\documentclass{peterlitsdoc}

\usepackage{hyperref}
\hypersetup{hidelinks}

\def\go{GoLang}

\title{East - 基于LALR、链表和tf-idf的单词搜索}
\author{周泓余%
    \thanks{邮箱:\href{mailto:peterlitszo@outlook.com}
                       {<peterlitszo@outlook.com>}}%
    }

\begin{document}

\maketitle
\tableofcontents
\newpage

%%%%%%%%%%%%%%%%%%%%%%%%%%%%%%%%%%%%%%%%%%%%%%%%%%%%%%%%%%%%%%%

\section{版本}

当前版本为:
\begin{lstlisting}
version 0.4.2
\end{lstlisting}

%%%%%%%%%%%%%%%%%%%%%%%%%%%%%%%%%%%%%%%%%%%%%%%%%%%%%%%%%%%%%%%

\section{部署与使用}

\subsection{部署}

请按照:
\begin{lstlisting}
$ go get -u -v github.com/PeterlitsZo/argparse
$ go build -o main ./src
\end{lstlisting}

或者:(如果在Linux环境下推荐使用)
\begin{lstlisting}
$ make init
$ make
\end{lstlisting}

进行编译。因为本软件是基于Goolge开发的开源语言\go 编写而成,使用
前需要安装\go 的编译器。\go 语言是一门比较小众的语言,所以如果没有
该编译器的话,还请进入\url{https://golang.org/dl/}下载。

输入命令\verb|./main version|来检查是否部署成功。如果部署成功,则会出现
如下的信息:
\begin{lstlisting}
$ ./main version
East version 0.4.x
$
\end{lstlisting}

%%%%%%%%%%%%%%%%%%%%%%%%%%%%%%%%%%%%%%%%%%%%%%%%%%%%%%%%%%%%%%%

\subsection{入门}

\subsubsection{主要命令}

编译完成后,输入下命令可以得到主要用法:
\begin{lstlisting}
$ ./main --help
usage: East <Command> [-h|--help]
description:
  sreach engine on file system
commands:
  run          the command to get the ID list (see README.pdf)
  mkindex      use this flag to make index named 'index.dict'
  interactive  make Self under the Interactive mode
  version      Show East's Version
arguments:
  -h, --help  Print help information
$
\end{lstlisting}

通过\verb|mkindex|命令可以为\verb|dirpath|文件夹(默认值为``input'',但可以根
据\verb|--dirpath|来进行修改)生成引索,然后通过\verb|--useindex|来让其
它命令使用引索文件。如果没有\verb|--useindex|命令,则会动态遍历
文件夹,分词,以哈希表加链表作为数据结构(当然,使用引索文件的
话,虽然不需要遍历分词,但是还是需要抽象成数据结构)。

最后使用\verb|run|命令或者进入\verb|interactive|模式来完成主要目标:检索。
语法见后文。

\subsubsection{检索语言}

这一次选择使用\verb|goyacc|工具来生成相应的LALR解析器源码,我首先实现了
一个分词器,然后编写BNF式的语法规则,构建了解析器\verb|parse.y|。

语法规则如下:
\begin{lstlisting}
<ast>          ::= SREACH <sreach_word>
                |  LIST
<sreach_word>  ::= <expr> OR <sreach_word>
                |  <expr>
<expr>         ::= <atom> AND <expr>
                |  <atom>
<atom>         ::= NOT STR
                |  STR
                |  NOT '(' <sreach_word> ')'
                |  '(' <sreach_word> ')'
\end{lstlisting}

实例:
\begin{lstlisting}
1. sreach 'in' AND NOT 'in' OR 'her'
2. sreach "'" or not '"'
3. sreach 'outside' && !'inside'
4. sreach (('that' or !'that') and 'that')
\end{lstlisting}

如实例所述,STR字符串是由单引号或者双引号包裹而成,支持为单引号或
者双引号转义(其实在命令行结构下本身就需要对引号进行转义,所以难免
会发生二次转义,比如\verb|--command="'\\\''"|这样比较丑陋的语法)。

不过现在甚至可以在交互模式下使用了。使用\verb|-interactive|进入交互模式
,然后在交互下输入命令,就算是同时使用单引号和双引号也轻轻松松、再
也不用担心二次转义啦(看到就是赚到)。

而AND,OR,NOT不仅仅支持单词(忽略大小写)样式,还支持C式的逻辑
运算符,也就是说\verb|&&|,\verb-||-和\verb|!|这三个传统命令。

现在还支持括号来实现更加高级的操作了。

%%%%%%%%%%%%%%%%%%%%%%%%%%%%%%%%%%%%%%%%%%%%%%%%%%%%%%%%%%%%%%%

\section{设计与框架}

\subsection{源代码组织}

总体来说,所有源代码都在\verb|src|文件夹下,为:
\begin{lstlisting}
./src
|-- argparse
|   `-- argparse.go
|-- list
|   `-- list.go
|-- logic
|   `-- logic.go
|-- main.go
|-- parse
|   |-- parse.go
|   `-- parse.y
`-- units
    |-- file.go
    |-- split.go
    |-- version.go
    `-- wordmap.go

5 directories, 10 files
\end{lstlisting}

按照模块化结构,所有的源代码都分工合作,各司其职。分别对他们进行
短暂描述有:

\begin{description}[labelindent=\parindent]
    \item[argparse] 解析命令行命令,提供usage提示。
    \item[list] 提供有序链表的接口。
    \item[logic] 实现了底层,如根据抽象出的语法树处理业务
        进行搜索处理。
    \item[parse] 讲前端的命令处理为抽象语法树。
    \item[units] 杂项,其他。包含:
        \begin{description}
            \item[file] 提供处理文件的接口。
            \item[spilt] 提供文本分词的接口,用来见文件转
                换为数据结构,或者说是index文件的第一步。
            \item[version] 提供版本信息。
            \item[wordmap] 抽象出的数据结构。
        \end{description}
\end{description}

\subsection{整体逻辑组织}

整体上来说逻辑分层,分别分为I/O层,Mid层\footnote{用来处理大部分
逻辑,包括,处理从I/O中传入的数据,使用Low层作为工具进行逻辑处理,
返回Low层数据结构},Low层\footnote{定义和抽象绝大多数的数据结构。}%
和Sys层\footnote{如内存管理等等。使用\go 语言操控,并没有实际接触。}。
大体来说,整体的逻辑组织结构如图\ref{struct}。

\begin{figure}[H]
\centering
\tikzstyle{outer} = [rectangle, draw=black!50, fill=black!20, thick,
                     minimum size=6mm, minimum width=10em, rounded corners]
\tikzstyle{inter} = [rectangle, draw=blue!50, fill=blue!20, thick,
                     minimum size=6mm, minimum width=10em, rounded corners]
\tikzstyle{labal} = [auto, swap, align = left]
\tikzstyle{biglabel} = [align = right, left]
\tikzstyle{bigline} = [very thick, black!50]
\setlength{\fboxsep}{1em}
\footnotesize
\fbox{%
\begin{tikzpicture}
    \setlength{\baselineskip}{0.8\baselineskip}
    \node (command)   at (5.5em, 0)          [outer, minimum width = 21em] {Input};
    \node (screen)    at (22em, 0)           [outer] {Output};
    \node (argparse)  at (5.5em, -3.5em)     [inter, minimum width = 21em] {ArgParser}
          edge [<-] node[labal] {East argument commands} (command);
    \node (parser)    at (0, -8em)           [inter] {Parser}
          edge [<-] node[labal] {East standard\\commands} (argparse.south -| parser);
    \node (logicer)   at (0, -11.5em)        [inter] {Logicer}
          edge [<-] node[labal] {AST} (parser);
    \node (low level) at (11em, -15em)       [inter, minimum width = 32em]
                                             {Data Struct, Help function/method, Etc.}
          edge [<-] node[labal] {East standard\\function/method}
                                             (argparse.south -| low level)
          (low level.north -| logicer) edge [<->] node[labal] {East data struct} (logicer)
          (low level.north -| screen) edge [->] node[labal] {Output\\string} (screen);
    \node (system)    at (11em, -18.5em)     [outer, minimum width = 32em] {System}
          edge [<->] node[labal] {\go} (low level);

    \node (IO)  at (-7.5em, 0)       [biglabel] {I/O};
    \node (Mid) at (-7.5em, -7.5em)  [biglabel] {Mid};
    \node (Low) at (-7.5em, -15em)   [biglabel] {Low};
    \node (Sys) at (-7.5em, -18.5em) [biglabel] {Sys};

    \draw [bigline] (command.north -| -7em, 0)   -- (command.south -| -7em, 0);
    \draw [bigline, blue!50] (argparse.north -| -7em, 0)  -- (logicer.south -| -7em, 0);
    \draw [bigline, blue!50] (low level.north -| -7em, 0) -- (low level.south -| -7em, 0);
    \draw [bigline] (system.north -| -7em, 0)    -- (system.south -| -7em, 0);
\end{tikzpicture}
}
\caption{the struct of East}
\label{struct}
\end{figure}

%%%%%%%%%%%%%%%%%%%%%%%%%%%%%%%%%%%%%%%%%%%%%%%%%%%%%%%%%%%%%%%

\section{示例}

[ 需要更新 ]

\begin{lstlisting}
$ # ---[ build itself ]------------------------------------------
$
$ go build -o main ./src
$ ./main --help
usage: East [-h|--help] [-d|--dirpath "<value>"] [-c|--command "<value>"]
            [-m|--mkindex] [-u|--useindex] [-i|--interactive]

            sreach engine on file system

Arguments:

  -h  --help         Print help information
  -d  --dirpath      the input files' folder path. Default: input
  -c  --command      the command to get the ID list (see README.pdf). Default: 
  -m  --mkindex      use this flag to make index named 'index.dict'
  -u  --useindex     use file 'index.dict' to find result
  -i  --interactive  make self under the interactive mode
$
$ # ---[ a little test using `-command` ]------------------------
$
$ ./main --command="'in' || 'not'"
result: [ d01.txt, d10.txt, d02.txt, d03.txt, d04.txt, d06.txt, d07.txt, d08.txt, d09.txt ]
$
$ # -[ use flag `-mkindex` and `useindex` to hold the result ]---
$
$ ls
README.md
README.pdf
input
main
make.sh
src
$
$ ./main --mkindex
$
$ ls # it will make a new file named index.dict
README.md
README.pdf
index.dict
input
main
make.sh
src
$
$ ./main --useindex --command="not 'in'"
result: [ d5.txt ]
$
$ ---[ the interactive mode ]------------------------------------
$
$ ./main --useindex --interactive
Enter `quit` for quit
copyleft (C) Peterlits Zo <peterlitszo@outlook.com>
Github: github.com/PeterlitsZo

Command > "that" and 'peter'
result: [ ]
Command > 'that' and ('peter' or !'peter')
result: [ d06.txt, d07.txt ]
Command > quit
$ 
\end{lstlisting}

%%%%%%%%%%%%%%%%%%%%%%%%%%%%%%%%%%%%%%%%%%%%%%%%%%%%%%%%%%%%%%%

\section{TODO、版本历史等辅助信息}

\subsection{ToDO}
\begin{plttodoenv}{1}
    \t[ ] 有什么办法可以来一键安装\verb|GoLang|的依赖吗?
    \t[ ] 更加详细的,函数式的结构。
    \t[x] 指出list来列出所有文件。
\end{plttodoenv}
\bigskip

关于一些长期计划列在下方:

为了实现命令list,我们需要:更聪明的解释AST(8),强化logicer的
返回(1)。
\pltbar{命令list}{1}{9}

\subsection{版本历史}

\def\hline{\begingroup
    \noindent\pltgray\rule{\textwidth}{0.5pt}%
    \endgroup}

0.4.2: 全面更新了文档。使用\LaTeX{}而非\verb|Markdown|来编写文档。

0.4.1: 重构,并让\verb|interactive|成为默认选项。

0.4.0: 使用\verb|sub-command|。

\hline

0.3.0: 使用前置命令以支持更多的操作,目前支持命令\verb|sreach|。

\hline

0.2.4: 使用\verb|Peterlits/argparse|替代原作者的库,以获得更好的usage输
出(不过如果原作者如果接受了我的pull request的话,那么其实可能又
会换回来)。

\hline

before: 未记录。

\end{document}

